%% LyX 2.4.0~RC3 created this file.  For more info, see https://www.lyx.org/.
%% Do not edit unless you really know what you are doing.
\documentclass[spanish]{article}
\renewcommand{\familydefault}{\sfdefault}
\usepackage[T1]{fontenc}
\usepackage[utf8]{inputenc}
\usepackage{babel}
\addto\shorthandsspanish{\spanishdeactivate{~<>}}
\deactivatequoting

\usepackage[bookmarks=true,bookmarksnumbered=false,bookmarksopen=false,
 breaklinks=false,pdfborder={0 0 1},backref=false,colorlinks=false]
 {hyperref}
\hypersetup{pdftitle={Actividad 1 Cálculo Numérico},
 pdfauthor={Jesús María Mora Mur},
 pdfsubject={Cálculo Numérico}}
\begin{document}
\vfill{}

\title{Actividad: métodos numéricos en C++}
\author{Jesús María Mora Mur}
\date{\today}

\maketitle
\vfill{}

\pagebreak{}

\tableofcontents{}

\pagebreak{}

\section{Descripción de la actividad.}

En la presente actividad se han trabajado los métodos de Muller y
Ridders para la resolución de ecuaciones. Dichos métodos son numéricos
y utilizan la interpolación cuadrática y exponencial, respectivamente,
para posibilitar la resolución de la ecuación siguiente:

\[
f(x)=\mathrm{e}^{0.75\cdot x}-3\cdot\sin\left(1.25\cdot x\right)
\]

Se evaluará la rapidez de los métodos en base a las iteraciones que
realizan hasta llegar a la solución con una precisión de $10^{-6}$unidades.
Se exponen a continuación los parámetros necesarios para que los métodos
realicen correctamente la tarea encomendada.

\subsection{Método de Muller.}

El método de Muller pretende interpolar la función a una parábola
en un entorno localizado de una función $f(x)$. Dados dos puntos
extremos y su punto medio, es posible obtener una parábola que se
acerque a la función. Encontrando las soluciones a la anulación de
la parábola conseguimos una aproximación. En función de en qué subintervalo
se encuentre la solución, se escoge para conseguir acotar más la solución.
El método converge, pero de manera lenta.

\subsection{Método de Ridders.}

El método de Ridders pretende aproximar la función a una exponencial
a la que se le aplica el método \emph{regula falsi}.
\end{document}
